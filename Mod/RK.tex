\documentclass[14pt,a4paper,oneside]{extarticle}
\usepackage{amsmath}\usepackage{amsfonts}\usepackage{amssymb}\usepackage{pgfplots}
\usepackage{setspace}\usepackage{mathtools}\usepackage{tikz}\usetikzlibrary{arrows}
\usepackage[left=3cm,right=1cm, top=2cm,bottom=2cm,bindingoffset=0cm]{geometry}
\usepackage[T2A]{fontenc}
\usepackage[utf8]{inputenc}
\usepackage[russian,english]{babel}
\usepackage{enumitem}
\makeatletter\AddEnumerateCounter{\asbuk}{\russian@alph}{щ}\makeatother
\DeclareMathOperator{\sinc}{sinc}\pgfplotsset{compat=1.18}\usepackage{setspace}
% \onehalfspacing
\newcommand{\tb}[1]{\textbf{#1}}
\newcommand{\pict}{\[\tb{Картинка}\]}
\newcommand{\fo}{\[\tb{Формула}\]}
\usepackage{tikz}
\usetikzlibrary{shapes.geometric,shapes.arrows,decorations.pathmorphing}
\usetikzlibrary{matrix,chains,scopes,positioning,arrows,fit, automata}
\usepackage{bookmark}
\usepackage{blindtext}
% \setcounter{secnumdepth}{0}
\usepackage{graphicx}
\graphicspath{ {./imgs/} }
\usepackage{relsize}
\usepackage{indentfirst}
\usepackage{float}
% \usepackage{sectsty}

% \sectionfont{\large}
% \sectionfont{\normalsize}
% \sectionfont{\normalsize}
% \paragraphfont{\normalsize}

\usepackage{amsthm}
\newtheorem*{theorem}{Теорема}
% \makeatletter
% \renewcommand{\@seccntformat}[1]{\csname the#1\endcsname.\quad}
% \makeatother
\hypersetup{%
    pdfborder = {0 0 0}
}

\usepackage{ragged2e}

% \setlength{\RaggedRightParindent}{\parindent}

\newlength{\normalparindent}
\setlength{\normalparindent}{\parindent}
\raggedright
\setlength{\parindent}{\normalparindent}

\newcommand{\pic}[3]{
	\begin{figure}[#1]
		\begin{center}
			\includegraphics[width=#2]{imgs/#3}
		\end{center}
		% \vspace{-5mm}\caption{#4}\label{fig:#5}
	\end{figure}
}

\begin{document}

% \RaggedRight

% \emergencystretch 3em

\begin{titlepage}
    \begin{center}
        \vspace*{1cm}

        \Huge
        \textbf{Моделирование радиотехнических систем}
        \vspace{1.5cm}

        \vfill
        \Huge
        \textbf{РК}\\
        \vspace{0.5cm}
        \LARGE
        % Преподаватель:\\Постникова Елена Сергеевна

        % postnikova.el@bmstu.ru

        % postnikova.el@yandex.ru

        % 8-916-733-16-46


        \vspace{1.5cm}

        % Кафедра: ИБМ-2

        % Каб. 506

        \vfill

        \LARGE
        Чекановский Сергей\\
        РЛ1-112

        \vspace{0.8cm}

        % \includegraphics[width=0.4\textwidth]{university}

        \Large
        МГТУ им. Н.Э. Баумана

    \end{center}
\end{titlepage}

\tableofcontents

\clearpage

\section{Рекуррентный алгоритм моделирования экспоненциальной функции}

\[u(t)=e^{at}\]

В дискретном представлении экспоненциальная функция будет иметь вид:

\begin{gather*}
    u[n]=e^{a\cdot n\Delta t}\\
    e^{an\Delta t}=e^{a}e^{a(n-1)\Delta t}\\
    u[n]=e^{an\Delta t}\\
    e^{an\Delta t}=e^{a\Delta t}e^{a(n-1)\Delta t}\\
    u[n]=e^{a\Delta t}\cdot u[n-1]
\end{gather*}

\end{document}
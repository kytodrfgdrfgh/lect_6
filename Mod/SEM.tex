\documentclass[14pt,a4paper,oneside]{extarticle}
\usepackage{amsmath}\usepackage{amsfonts}\usepackage{amssymb}\usepackage{pgfplots}
\usepackage{setspace}\usepackage{mathtools}\usepackage{tikz}\usetikzlibrary{arrows}
\usepackage[left=3cm,right=1cm, top=2cm,bottom=2cm,bindingoffset=0cm]{geometry}
\usepackage[T2A]{fontenc}
\usepackage[utf8]{inputenc}
\usepackage[russian,english]{babel}
\usepackage{enumitem}
\makeatletter\AddEnumerateCounter{\asbuk}{\russian@alph}{щ}\makeatother
\DeclareMathOperator{\sinc}{sinc}\pgfplotsset{compat=1.18}\usepackage{setspace}
% \onehalfspacing
\newcommand{\tb}[1]{\textbf{#1}}
\newcommand{\pict}{\[\tb{Картинка}\]}
\newcommand{\fo}{\[\tb{Формула}\]}
\usepackage{tikz}
\usetikzlibrary{shapes.geometric,shapes.arrows,decorations.pathmorphing}
\usetikzlibrary{matrix,chains,scopes,positioning,arrows,fit, automata}
\usepackage{bookmark}
\usepackage{blindtext}
% \setcounter{secnumdepth}{0}
\usepackage{graphicx}
\graphicspath{ {./imgs/} }
\usepackage{relsize}
\usepackage{indentfirst}
\usepackage{float}
% \usepackage{sectsty}

% \sectionfont{\large}
% \subsectionfont{\normalsize}
% \subsubsectionfont{\normalsize}
% \paragraphfont{\normalsize}

\usepackage{amsthm}
\newtheorem*{theorem}{Теорема}
% \makeatletter
% \renewcommand{\@seccntformat}[1]{\csname the#1\endcsname.\quad}
% \makeatother
\hypersetup{%
    pdfborder = {0 0 0}
}

\usepackage{ragged2e}

% \setlength{\RaggedRightParindent}{\parindent}

\newcommand{\pic}[3]{
	\begin{figure}[#1]
		\begin{center}
			\includegraphics[width=#2]{imgs/#3}
		\end{center}
		% \vspace{-5mm}\caption{#4}\label{fig:#5}
	\end{figure}
}

\newlength{\normalparindent}
\setlength{\normalparindent}{\parindent}
\raggedright
\setlength{\parindent}{\normalparindent}

\begin{document}

% \RaggedRight

% \emergencystretch 3em

\begin{titlepage}
    \begin{center}
        \vspace*{1cm}

        \Huge
        \textbf{Моделирование радиотехнических систем}



        \vspace{1.5cm}

        \vfill
        \Huge
        \textbf{Семинары}\\
        \vspace{0.5cm}
        \LARGE
        Преподаватель: Исаев И.Д.
        \vfill

        \LARGE
        Чекановский Сергей\\
        РЛ1-112

        \vspace{0.8cm}

        % \includegraphics[width=0.4\textwidth]{university}

        \Large
        МГТУ им. Н.Э. Баумана

    \end{center}
\end{titlepage}

\tableofcontents

\clearpage

\section{Семинар 1}

\subsection{Моделирование детерминированных функций}

\begin{gather*}
    \sin(at)=u_1(t)\\
    \cos(at)=u_2(t)\\
    u_1[n]=\sin(an\Delta t)=\sin(a(n+1-1)\Delta t)=\\
    =\sin(a(n+1))
\end{gather*}

пропуск

\begin{gather*}
    =u_1[n-1]\cdot c_1+u_2[n-1]\cdot c_2\\
    u_1[0]=0\\
    u_2[0]=1\\
    u_2[n]=\cos(a\Delta t(n-1))\cos(a\Delta t)-\sin(a\Delta t(n-1))\sin(a\Delta t)=\\
    =u_2[n-1]c_1-u_1[n-1]c_2\\
    u_1[0]=0\\
    u_2[0]=1
\end{gather*}

\subsection{Моделирование случайных чисел в диапазоне 0...1}

\begin{gather*}
    m_x=\int\displaylimits_{-\infty}^{+\infty} x\cdot w(x)dx\\
    m=\int\displaylimits_{a}^{b}x\frac{1}{b-a}dx=\frac{1}{b-a}\frac{x^2}{2}|_a^b=\frac{b^2-a^2}{2(b-a)}=\frac{a+b}{2}
\end{gather*}

\begin{gather*}
    F(X)=\int\displaylimits_{a}^{x}w(y)dy\\
    w(y)=\frac{dF(y)}{dy}\\
\end{gather*}

\begin{enumerate}
    \item $\int\displaylimits_{-\infty}^{+\infty}w(y)dy=1$
    \item $w(y)\geq 0$
    \item $F(y)$ - неубывающая
    \item $0\leq F(y)\leq 1$
\end{enumerate}

\subsection{Метод преобразования, обратного функции распределения}

\begin{gather*}
    x=a+(b-a)\underbrace{F(y)}_{CB \forall 3P}
\end{gather*}

x - равномерная случайная величина в $[a,b]$

\begin{gather*}
    w(y)=w(x)\left\lvert\frac{dx}{dy} \right\rvert *\\
    \frac{dx}{dy}=(b-a)\frac{dF(y)}{dy}=(b-a)w(y)\\
    \frac{w(y)}{w(x)}=(b-a)w(y)\Longrightarrow w(x) = \frac{1}{b-a}
\end{gather*}

\begin{gather*}
    w(y)=\frac{y}{\sigma^2}e^{-\frac{y^2}{2\sigma^2}}\\
    y\in[0;+\infty)\\
    F(y)=\int\displaylimits_{0}^{y}\frac{x}{\sigma^2}e^{-\frac{x^2}{2\sigma^2}}dx=-\int\displaylimits_{0}^{y}e^{-\frac{x^2}{2\sigma^2}}d(-\frac{x^2}{2\sigma^2})=e^{-\frac{x^2}{2\sigma^2}}\lvert_y^0=1-e^{-\frac{y^2}{2\sigma^2}}=x\\
    1-x=e^{-\frac{y^2}{2\sigma^2}}\\
    -\frac{y^2}{2\sigma^2}=\ln(1-x)\\
    y=\sqrt{-2\sigma^2\cdot\ln(1-x)}=\sigma\sqrt{-2\ln(x)}\\
\end{gather*}

\begin{gather*}
    w(y)=\lambda e^{-\lambda y}\\
    F(y)=\int\displaylimits_{0}^{y}\lambda e^{-\lambda x}dx=\frac{1}{\lambda}\int\displaylimits_{0}^{y}\lambda e^{-\lambda x}d(-\lambda x)=\\
    =-\frac{1}{\lambda}\cdot\lambda\cdot e^{-\lambda x}|_0^y=-(e^{-\lambda y}-e^0)=1-e^{-\lambda y}=x \\
    1-e^{-\lambda y}=x\Longrightarrow e^{-\lambda y}=1-x\Longrightarrow -\lambda y= C_n(1-x)\Longrightarrow\\
    \Longrightarrow y=-\frac{1}{\lambda}\ln(1-x)=-\frac{1}{\lambda}\ln(x)
\end{gather*}

\begin{gather*}
    w(y)=\frac{1}{\pi a\sqrt{1-\left(\frac{y}{a}\right)^2 }}\\
    F(y)=\int\displaylimits_{-a}^{y}\frac{1}{\pi a\sqrt{1-\left(\frac{y}{a}\right)^2 }}dx=\frac{a}{\pi a}\int\displaylimits_{-a}^{y}\frac{1}{\sqrt{1-\left(\frac{y}{a}\right)^2 }}d\frac{x}{a}=\\
    =\frac{1}{\pi}\arcsin\left(\frac{x}{a}\right)|_{-a}^{y}=\frac{1}{\pi}\left(\arcsin\frac{y}{a}-\arcsin(-1)\right)=\\
    =\frac{1}{\pi}\left(\arcsin\left(\frac{y}{a}-\frac{\pi}{2}\right)\right)=\frac{1}{\pi}\arcsin\frac{y}{a}+\frac{1}{2}=x\\
    y=a\cdot\sin\left(\underbrace{\pi\left(x-\frac{1}{2}\right) }_{-1\ldots1} \right),\text{ } -\frac{\pi}{2}\ldots\frac{\pi}{2}
\end{gather*}

Нельзя применить когда функция аналитически не выражается.

\subsection{Логнормальный}

Пропуск см фото

\end{document}
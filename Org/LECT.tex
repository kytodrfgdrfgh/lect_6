\documentclass[14pt,a4paper,oneside]{extarticle}
\usepackage{amsmath}\usepackage{amsfonts}\usepackage{amssymb}\usepackage{pgfplots}
\usepackage{setspace}\usepackage{mathtools}\usepackage{tikz}\usetikzlibrary{arrows}
\usepackage[left=3cm,right=1cm, top=2cm,bottom=2cm,bindingoffset=0cm]{geometry}
\usepackage[T2A]{fontenc}
\usepackage[utf8]{inputenc}
\usepackage[russian,english]{babel}
\usepackage{enumitem}
\makeatletter\AddEnumerateCounter{\asbuk}{\russian@alph}{щ}\makeatother
\DeclareMathOperator{\sinc}{sinc}\pgfplotsset{compat=1.18}\usepackage{setspace}
% \onehalfspacing
\newcommand{\tb}[1]{\textbf{#1}}
\newcommand{\pict}{\[\tb{Картинка}\]}
\newcommand{\fo}{\[\tb{Формула}\]}
\usepackage{tikz}
\usetikzlibrary{shapes.geometric,shapes.arrows,decorations.pathmorphing}
\usetikzlibrary{matrix,chains,scopes,positioning,arrows,fit, automata}
\usepackage{bookmark}
\usepackage{blindtext}
% \setcounter{secnumdepth}{0}
\usepackage{graphicx}
\graphicspath{ {./imgs/} }
\usepackage{relsize}
\usepackage{indentfirst}
\usepackage{float}
% \usepackage{sectsty}

% \sectionfont{\large}
% \subsectionfont{\normalsize}
% \subsubsectionfont{\normalsize}
% \paragraphfont{\normalsize}

\usepackage{amsthm}
\newtheorem*{theorem}{Теорема}
% \makeatletter
% \renewcommand{\@seccntformat}[1]{\csname the#1\endcsname.\quad}
% \makeatother
\hypersetup{%
    pdfborder = {0 0 0}
}

\usepackage{ragged2e}

% \setlength{\RaggedRightParindent}{\parindent}

\newlength{\normalparindent}
\setlength{\normalparindent}{\parindent}
\raggedright
\setlength{\parindent}{\normalparindent}

\newcommand{\pic}[3]{
	\begin{figure}[#1]
		\begin{center}
			\includegraphics[width=#2]{imgs/#3}
		\end{center}
		% \vspace{-5mm}\caption{#4}\label{fig:#5}
	\end{figure}
}

\begin{document}

% \RaggedRight

% \emergencystretch 3em

\begin{titlepage}
    \begin{center}
        \vspace*{1cm}

        \Huge
        \textbf{Организация и планирование производства}
        \vspace{1.5cm}

        \vfill
        \Huge
        \textbf{Лекции}\\
        \vspace{0.5cm}
        \LARGE
        Преподаватель:\\Постникова Елена Сергеевна

        postnikova.el@bmstu.ru

        postnikova.el@yandex.ru

        8-916-733-16-46


        \vspace{1.5cm}

        Кафедра: ИБМ-2

        Каб. 506

        \vfill

        \LARGE
        Чекановский Сергей\\
        РЛ1-112

        \vspace{0.8cm}

        % \includegraphics[width=0.4\textwidth]{university}

        \Large
        МГТУ им. Н.Э. Баумана

    \end{center}
\end{titlepage}

\tableofcontents

\clearpage

\section{Лекция 1}

\subsection{Историческая справка}

Советов и Платонов разработали русскую школу обучения механическим мастерствам:

\begin{itemize}
    \item Токарные искусства по дереву
    \item Столярные
    \item Кузнечные
    \item Токарные по металлу
    \item Слесарные
    \item Модельные
\end{itemize}

В качестве пособий - модели реальных инструментов в увеличенном масштабе и коллекции измерительных инструментов с указанием областей и способов их применения.

Заслуга Советова и Платонова заключается в том что они смогли выделить элементарные последовательности работ для каждого вида мастерства.

Не выделили технологическую операцию - не хватило одного шага для того чтобы стать первыми авторами организации производства.

Фредерик Тейлор:

\begin{itemize}
    \item Отделил подготовительные работы от выполнения операций
    \item Дифференцировал процесс труда - как правило 1 операция на рабочего
    \item Ввел хронометраж для устранения лишних приемов операции
    \item Интенсифицировал труд рабочих, введя сдельно-дифференциальную оплату труда. При выплонении нормы - оплана по высокой шкале, при невыполнении - по низкой. Норма устанавивалась на основе труда лучших рабочих.
    \item Ввел аппарат функциональных руководитлей, мастеров и инструкторов
    \item Ввел конвейер
\end{itemize}

Хронометраж - метод нормирования оперативного времени выполнения операции.

\[t_\text{оп}=t_o+t_e\]

Конвейер Форда - идея Тейлора.

Генри Форд:

\begin{itemize}
    \item Максимально разделил труд рабочих, в результате чего все операции могли выполняться рабочими с низкой квалификацией при исключительно напряженном темпе работы, который соблюдался механическими регуляторами ритма.
    \item Организовал массовое производства и предметно замкнутые участки и линии с прямоточным характером движения.
    \item Были стандартизованы все факторы производства такие как сырье, оборудование, инструмент, технологические режимы, трудовые приемы.
\end{itemize}

Карл Адамецкий разработал движение деталей по операциям и формулы расчета производственного цикла.

Иейс - учет человеческого фактора, влияния условия труда на производительность.

Д. Макгрегор - теория человеческих отношений.

Уолкер, Келли, Малькольм - СПУ. Для планирования инновационных процессов, построена на вероятностных оценках времени работ. Можно использовать и в производстве.

Семенов - аналитическое нормирование труда.

Каценбоген - орагнизация поточного производства.

Орентлихтер, Иоффе - справочные нормативы.

Митрофанов - организация группового производства.

\subsection{Органиазция производственных процессов}

\subsubsection{Общая и производственная структура предприятия}

\pic{H}{\textwidth}{1}

Состав производственных звеньев, органов управленияи организации обслуживания работников, их количество, их количество, величина и соотношение занятых ими площадей, численности работников и пропускной способности представляют общую структуру предприятия.

Производственные звенья: цеха и участки, где изготавливается основная продукция.

Вспомогательное производство: техоснастка, тара.

Хозяйство и цеха оказывают услуги для основных и вспомогательных звеньев (например транспорт).

Отличия цеха от хозяйства - наличие хотя бы одного станка (участка производства). Если оказываем только услуги - хозяйство.

Структурные элементы предприятия создаются в соответствии с расчленением процесса производственного процесса на частичные производственные процессы и частичные операции.

Производственная структура зависит от следующих произыводственных факторов:

\begin{itemize}
    \item От конструктивных особенностей и методов изготовления 
    \item Обьема выпуска продукции 
    \item Уровня и формы специализации и кооперирования предприятия
\end{itemize}

Конструктивные особенности определяют состав и характер производственных процессов, следовательно состав цехов и производсственных участков.

Объем производства так же влияет на выбор метода изготовления.

Обьем производства оказывает влияние на дифференциацию производственной структуры и на сложность связей между цехами, на размер и количество цехов.

Специализация - узкая или широкая.

Уровень специализации - высокий или низкий.

\pic{H}{\textwidth/2}{9}

\begin{enumerate}
    \item Предприятие полного технологического цикла 
    \item Предприятие подетальной специализации
    \item Сборочное предприятие
    \item Заготовительное предприятие 
    \item Механо-сборочное предприятие
\end{enumerate}

Наиболее высокий уровень специализации - при широкой номенклатуре.

Первичным звеном производственной структуры предприятия является рабочее место - часть производственной площади, оснащенное соответствующим оборудованием и оснащением, на которой рабочий или группа рабочих выполняют отдельную операцию по изготовлению продукции или обслуживанию процесса производства.

Рабочие места связанные между собой выполнением определенной части производственного процесса или изготовлением какого либо вида продукции обьединяются в производственные участки основного производства.

Многие мелкие и средние предприятия имеют цеховую структуру (состоят из производственных участков), на крупномм и среднем предприятии участки связанные между собой в процессе производства и нуждающиеся в едином управлении обьединяются в цеха.

Цех - организационно и технологически обособленное звено предприятия, выполняющая определенную часть производственного процесса, либо изготавливающее какой либо вид продукции предприятия.

В соответствии с назначением производственных процессов цеха делятся на основные, вспомогательные и обслуживающие цеха и хозяйства.

В основных цехах выполняются производственные процессы связанные с изготовлением программной продукцией предприятия.

Цеха бывают заготовительные, обрабатывающие и сборочные.

Изготовительные: раскройный, литейный, кузнечно-штамповочный, кузнечно-прессовый.

Обрабатывающие: механические, цех покрытий, механосборочный

Сборочные: узловой, окончательной сборки.

Вспомогательные цеха: изготавливают продукцию потребляемую внутри предприятия. Пример: инструментальный, ремонтно-механический, электроремонтный.

Технологческая специализация: при такой специализации в цехах (участках) выполняют определенные технологические операции ... при этотм в цехе устанавливается однотипное оборудование (технологически однородные). Оборудование располагается по групповому признаку (технологической однородности).

Плюсы:

При небольшой разнообразности облегчается техобслуживание

Минусы:

Складываются сложные удлиненные маршруты движение предметов труда с неоднократным их возвращением в одни и те же цеха и участки, что приводит к увеличению длительности производственного цикла и затрат времени на транспортировку.

Технологическую специализацию используют в мелкосерийном и единичном производстве.

При предметной специализации цеха \dots

Преимущества:



\end{document}
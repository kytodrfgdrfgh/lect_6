\documentclass[14pt,a4paper,oneside]{extarticle}
\usepackage{amsmath}\usepackage{amsfonts}\usepackage{amssymb}\usepackage{pgfplots}
\usepackage{setspace}\usepackage{mathtools}\usepackage{tikz}\usetikzlibrary{arrows}
\usepackage[left=3cm,right=1cm, top=2cm,bottom=2cm,bindingoffset=0cm]{geometry}
\usepackage[T2A]{fontenc}
\usepackage[utf8]{inputenc}
\usepackage[russian,english]{babel}
\usepackage{enumitem}
\makeatletter\AddEnumerateCounter{\asbuk}{\russian@alph}{щ}\makeatother
\DeclareMathOperator{\sinc}{sinc}\pgfplotsset{compat=1.18}\usepackage{setspace}
% \onehalfspacing
\newcommand{\tb}[1]{\textbf{#1}}
\newcommand{\pict}{\[\tb{Картинка}\]}
\newcommand{\fo}{\[\tb{Формула}\]}
\usepackage{tikz}
\usetikzlibrary{shapes.geometric,shapes.arrows,decorations.pathmorphing}
\usetikzlibrary{matrix,chains,scopes,positioning,arrows,fit, automata}
\usepackage{bookmark}
\usepackage{blindtext}
% \setcounter{secnumdepth}{0}
\usepackage{graphicx}
\graphicspath{ {./imgs/} }
\usepackage{relsize}
\usepackage{indentfirst}
\usepackage{float}
% \usepackage{sectsty}

% \sectionfont{\large}
% \subsectionfont{\normalsize}
% \subsubsectionfont{\normalsize}
% \paragraphfont{\normalsize}

\usepackage{amsthm}
\newtheorem*{theorem}{Теорема}
% \makeatletter
% \renewcommand{\@seccntformat}[1]{\csname the#1\endcsname.\quad}
% \makeatother
\hypersetup{%
    pdfborder = {0 0 0}
}

\usepackage{ragged2e}

% \setlength{\RaggedRightParindent}{\parindent}

\newcommand{\pic}[3]{
	\begin{figure}[#1]
		\begin{center}
			\includegraphics[width=#2]{imgs/#3}
		\end{center}
		% \vspace{-5mm}\caption{#4}\label{fig:#5}
	\end{figure}
}

\newlength{\normalparindent}
\setlength{\normalparindent}{\parindent}
\raggedright
\setlength{\parindent}{\normalparindent}

\begin{document}

% \RaggedRight

% \emergencystretch 3em

\begin{titlepage}
    \begin{center}
        \vspace*{1cm}

        \Huge
        \textbf{Организация и планирование производства}
        \textbf{Семинары}

        \vspace{0.5cm}
        \LARGE
        Преподаватель: Постникова Елена Сергеевна

        postnikova.el@bmstu.ru

        postnikova.el@yandex.ru

        8-916-733-16-46


        \vspace{1.5cm}

        Кафедра: ИБМ-2

        Каб. 506

        \vfill

        Чекановский Сергей\\
        РЛ1-112

        \vspace{0.8cm}

        % \includegraphics[width=0.4\textwidth]{university}

        \Large
        МГТУ им. Н.Э. Баумана

    \end{center}
\end{titlepage}

\tableofcontents

\clearpage

\section{Семинар 1}

\textbf{ПРОПУСК}

% \begin{equation*}
%     \text{Производственный процесс}=\text{Технологический процесс}+\\
%     +\underbrace{$\text{транспорт. м-ду раб местами}$}
% \end{equation*}


\textbf{Технологический процесс} - это совокупность и последовательность выполнения технологических операций.

Верменными характеристиками простого производственного процесса являются производственный цикл и технологический цикл.

\[T_\text{пр}=T_\text{техн}+\sum_{i=1}^m t_\text{м.о. i}+T_e\]

Где m - число операций.

Предметы труда могут передаваться с одной операции на другую поштучно или партиями.

\textbf{Производственный цикл} - это интервал времени от запуска исходных предметов труда до выпуска готовой продукции.

\textbf{Технологический цикл} - суммарное время выплонения всех технологических операций над деталью или патрией деталей.

При поштучной длительность технологического цикла рассчитывается как сумма времен выполнений всех операций технологического процесса. Время выполнения операции над одним предметом труда:

\[T\text{техн}=\sum_{i=1}^{m}\frac{t_i}{c_i}\]

$t_i$ - норма времени выполнения операции - технически и экономически обоснованное минимальное время выполнения i-той операции над одним предметом труда.

$c_i$ - число рабочих мест.

Норма времени может быть определена как штучная норма времени (производственная) или штучно- калькуляционная (полная).

\begin{gather*}
        t_\text{шт}=t_\text{оп}+\sum t_\text{перерыв}=t_\text{осн приемов опер}+t_\text{вспом пр оп}+ t_\text{пер техобсл}+\\+t_\text{пер орг обсл}+t_\text{пер переналадки Технол Сист}+t_\text{пер отд и личной надобности рабочего}
\end{gather*}

\begin{gather*}
    t_\text{шт-к}=t_\text{шт}+\frac{T_\text{п-з}}{n}
\end{gather*}

n - размер патрии деталей

$T_\text{п-з}$ - Подготовительно-заключительное время

При партионной передаче длительность технологического цикла зависит от вида движения, партий предметов труда по операциям технологического процесса.

Виды движения: последовательные, параллельно-последовательные и параллельные.

\subsection{Последовательный вид движения}

\subsubsection{Технологический цикл при последовательном движении}

При последовательном движении каждая следующая операция наччинается после полного завершения предыдущей операции над всей обработочной партией.

Время выполнения i-ой операции над обработочной партией называется операционным циклом i-ой операции. Непрерывные операционные циклы можно рассчитать по формуле:

\begin{gather*}
    T_\text{оп i}=k\frac{t_i}{c_i}
\end{gather*}

\begin{gather*}
    T_\text{техн посл}=\sum_{i=1}^{m}T_\text{оп i}=\sum_{i=1}^{m}\frac{t_i}{c_i}=n(\ldots)=\text{результат (ед изм)}
\end{gather*}

\subsubsection{Производственный цикл при последовательном виде движения}

пропуск фото

\begin{gather*}
    T_\text{пр}=\frac{T_\text{техн}+\sum_{i=1}^{m}t_\text{м.о.}}{T_\text{см}\cdot f\cdot k_\text{реп}}\\
    k_\text{реп}=\frac{k_\text{раб дн}}{k_\text{кал дн}}
\end{gather*}

\end{document}
\documentclass[14pt,a4paper,oneside]{extarticle}
\usepackage{amsmath}\usepackage{amsfonts}\usepackage{amssymb}\usepackage{pgfplots}
\usepackage{setspace}\usepackage{mathtools}\usepackage{tikz}\usetikzlibrary{arrows}
\usepackage[left=3cm,right=1cm, top=2cm,bottom=2cm,bindingoffset=0cm]{geometry}
\usepackage[T2A]{fontenc}
\usepackage[utf8]{inputenc}
\usepackage[russian,english]{babel}
\usepackage{enumitem}
\makeatletter\AddEnumerateCounter{\asbuk}{\russian@alph}{щ}\makeatother
\DeclareMathOperator{\sinc}{sinc}\pgfplotsset{compat=1.18}\usepackage{setspace}
% \onehalfspacing
\newcommand{\tb}[1]{\textbf{#1}}
\newcommand{\pict}{\[\tb{Картинка}\]}
\newcommand{\fo}{\[\tb{Формула}\]}
\usepackage{tikz}
\usetikzlibrary{shapes.geometric,shapes.arrows,decorations.pathmorphing}
\usetikzlibrary{matrix,chains,scopes,positioning,arrows,fit, automata}
\usepackage{bookmark}
\usepackage{blindtext}
% \setcounter{secnumdepth}{0}
\usepackage{graphicx}
\graphicspath{ {./imgs/} }
\usepackage{relsize}
\usepackage{indentfirst}
\usepackage{float}
% \usepackage{sectsty}

% \sectionfont{\large}
% \sectionfont{\normalsize}
% \subsectionfont{\normalsize}
% \paragraphfont{\normalsize}

\usepackage{amsthm}
\newtheorem*{theorem}{Теорема}
% \makeatletter
% \renewcommand{\@seccntformat}[1]{\csname the#1\endcsname.\quad}
% \makeatother
\hypersetup{%
    pdfborder = {0 0 0}
}

\usepackage{ragged2e}

% \setlength{\RaggedRightParindent}{\parindent}

\newcommand{\pic}[3]{
	\begin{figure}[#1]
		\begin{center}
			\includegraphics[width=#2]{imgs/#3}
		\end{center}
		% \vspace{-5mm}\caption{#4}\label{fig:#5}
	\end{figure}
}

\newlength{\normalparindent}
\setlength{\normalparindent}{\parindent}
\raggedright
\setlength{\parindent}{\normalparindent}

\begin{document}

% \RaggedRight

% \emergencystretch 3em

\begin{titlepage}
    \begin{center}
        \vspace*{1cm}

        \Huge
        \textbf{Организация и планирование производства}
        \vspace{1.5cm}

        \vfill
        \Huge
        \textbf{Семинары}\\
        \vspace{0.5cm}
        \LARGE
        Преподаватель:\\Постникова Елена Сергеевна

        postnikova.el@bmstu.ru

        postnikova.el@yandex.ru

        8-916-733-16-46


        \vspace{1.5cm}

        Кафедра: ИБМ-2

        Каб. 506

        \vfill

        \LARGE
        Чекановский Сергей\\
        РЛ1-112

        \vspace{0.8cm}

        % \includegraphics[width=0.4\textwidth]{university}

        \Large
        МГТУ им. Н.Э. Баумана

    \end{center}
\end{titlepage}

\tableofcontents

\clearpage

\section{ДЗ}

1 лист - титул

2 лист - исходные данные, вариант

Расчетная часть 

Графическая часть: 3 графика технологических и 3 графика расчетной части.

2 задачи

\section{Организация простого производственного процесса}

\textbf{Производственный процесс} - совокупность действий людей и орудий труда, выполнающихся на предприятии для изготовления готовой продукции а так же естественных процессов протекающих под воздействием природных условий.

\textbf{Простой процесс} - процесс изготовления детали, отдельно взятый сборочный процесс узловой или окончательной сборки или другой процесс, основой которого являются технологические операции.
\\
Производственный процесс=Технологический процесс+Транспортировка между рабочими местами+Складирование около рабочего места+Выборочный контроль+Естественные процессы
\\

\textbf{Технологический процесс} - это совокупность и последовательность выполнения технологических операций.

Верменными характеристиками простого производственного процесса являются производственный цикл и технологический цикл.

\[T_\text{пр}=T_\text{техн}+\sum_{i=1}^m t_\text{м.о. i}+T_e\]

Где m - число операций.

Предметы труда могут передаваться с одной операции на другую поштучно или партиями.

\textbf{Производственный цикл} - это интервал времени от запуска исходных предметов труда до выпуска готовой продукции.

\textbf{Технологический цикл} - суммарное время выплонения всех технологических операций над деталью или патрией деталей.

При поштучной длительность технологического цикла рассчитывается как сумма времен выполнений всех операций технологического процесса. Время выполнения операции над одним предметом труда:

\[T\text{техн}=\sum_{i=1}^{m}\frac{t_i}{c_i}\]

$t_i$ - норма времени выполнения операции - технически и экономически обоснованное минимальное время выполнения i-той операции над одним предметом труда.

$c_i$ - число рабочих мест.

Норма времени может быть определена как штучная норма времени (производственная) или штучно-калькуляционная (полная).

\begin{gather*}
        t_\text{шт}=t_\text{оп}+\sum t_\text{перерыв}=t_\text{осн приемов опер}+t_\text{вспом пр оп}+ t_\text{пер техобсл}+\\+t_\text{пер орг обсл}+t_\text{пер переналадки Технол Сист}+t_\text{пер отд и личной надобности рабочего}
\end{gather*}

\begin{gather*}
    t_\text{шт-к}=t_\text{шт}+\frac{T_\text{п-з}}{n}
\end{gather*}

n - размер патрии деталей

$T_\text{п-з}$ - Подготовительно-заключительное время

При партионной передаче длительность технологического цикла зависит от вида движения, партий предметов труда по операциям технологического процесса.

Виды движения: последовательные, параллельно-последовательные и параллельные.

\section{Последовательный вид движения}

\subsection{Технологический цикл при последовательном движении}

При последовательном движении каждая следующая операция наччинается после полного завершения предыдущей операции над всей обработочной партией.

Время выполнения i-ой операции над обработочной партией называется операционным циклом i-ой операции. Непрерывные операционные циклы можно рассчитать по формуле:

\begin{gather*}
    T_\text{оп i}=n\frac{t_i}{c_i}
\end{gather*}

\begin{gather*}
    T_\text{техн посл}=\sum_{i=1}^{m}T_\text{оп i}=n\sum_{i=1}^{m}\frac{t_i}{c_i}=n(\ldots)=\text{результат (ед изм)}
\end{gather*}

\subsection{Производственный цикл при последовательном виде движения}

пропуск фото

\begin{gather*}
    T_\text{пр}=\frac{T_\text{техн}+\sum_{i=1}^{m}t_\text{м.о.}}{T_\text{см}\cdot f\cdot k_\text{реж}}\\
    k_\text{реп}=\frac{k_\text{раб дн}}{k_\text{кал дн}}
\end{gather*}

f - число смен

$k_\text{реж}$ -режим работы предприятия

\section{Последовательно-параллельный вид движения}


\begin{itemize}
    \item Обработочная партия деталей делится на более мелкие транспортные партии которые могут передаваться на следующую операцию не дожидаясь окончания обработки всей обработочной партии. Размер всех транспортных партий одинаковый и не меняется в процессе производства.
    \item Операционные циклы всех операций непрервыны. При этом на следующую операцию без перерыва пролеживания передается либо первая либо последняя транспортная партия, остальные транспортные партии пролеживают в ожидании когда их возьмут в работу.
\end{itemize}

\begin{enumerate}
    \item Если $\frac{t_i}{c_i}<\frac{t_{i+1}}{c_{i+1}}$, то на следующую операцию без перервыа пролеживания передается первая транспортная партия.
    \item Если $\frac{t_i}{c_i}\geq \frac{t_{i+1}}{c_{i+1}}$, то последняя транспротная партия.
\end{enumerate}

\subsection{Технологический цикл при параллельно-последовательном виде движения}

\[\frac{t_1}{c_1}<\frac{t_2}{c_2}>\frac{t_3}{c_3}>\frac{t_4}{c_4}\]
\[c_2=3\]

$n=100$ шт/парт

$n_{\text{тр}}=25$ шт/парт

график

\[k_\text{непр}=\frac{T_\text{техн}-\sum t_\text{ож}^{max}}{T_\text{техн}}\]

Перерывы ожидания суммируются без дублирования

график

\[\tau_{12}=n\frac{t_1}{c_1}-n_\text{тр}\frac{t_1}{c_1}=(n-n_\text{тр})\frac{t_1}{c_1}\]

\[\tau_{23}=(n-n_\text{тр})\frac{t_3}{c_3}\]
\[\tau_{34}=(n-n_\text{тр})\frac{t_4}{c_4}\]
\[\tau_{i,i+1}=(n-n_\text{тр})\frac{t_i}{c_i}\]

i - индекс операции, короткой при попарном сравнении i и i+1.

\[T_\text{техн п-п}=T_\text{техн посл}-\sum\tau_{i,i+1}=n\sum_{i=1}^m\frac{t_i}{c_i}-(n-n_\text{тр})\sum_{j=1}^{m-1}\left(\frac{t_j}{c_j}\right) \]

Суммируется на 1 значение меньше чем количество операций

\subsection{Производственный цикл при параллельно-последовательном виде движения}

график

\section{Параллельное движение}

Обработочные партии делятся на транспортные партии которые без пролеживания передаются на каждую следующую операцию, т.е. технологические циклы обработки всех транспортных партий непрерывны. Для каждой партии последовательный вид движения.

Операционный цикл операции максимальной длительности непрерывен.

\subsection{Технологический цикл при параллельном движении}

\begin{enumerate}
    \item Строим технологический цикл обработки первой транспортной партии
    \item Достраиваем операционный цикл операции максимальной длительности
    \item Достраиваем технологические циклы для всех транспортных партий кроме 1
\end{enumerate}

график

\[T_\text{техн п-п}=n_\text{тр}\sum_{i=1}^m\frac{t_i}{c_i}+(n-n_\text{тр})\sum_{j=1}^{m-1}\left(\frac{t}{c}\right)_{max} \]

\subsection{Производственный цикл при параллельном движении}

график

Ошибка построения:

\[\Delta_\text{техн}=\frac{\left\lvert T_\text{техн}^\text{расп}-T_\text{техн}^\text{граф}\right\rvert }{T_\text{техн}^\text{расп}}\cdot 100\]

- в минутах

\[\Delta_\text{пр}=\frac{\left\lvert T_\text{пр}^\text{расп}-T_\text{пр}^\text{граф}\right\rvert }{T_\text{пр}^\text{расп}}\cdot 100\]

- в календарных днях.

\end{document}